%=================================================================================
% PACKAGES
\usepackage{amsmath}
\usepackage{amssymb}
\usepackage{amsfonts}
\usepackage{wasysym} % for circles
\usepackage{graphicx}
\usepackage[utf8]{inputenc}
%\usepackage[usenames,dvipsnames]{color}
\usepackage{textcomp}
\usepackage{alltt}
\usepackage{subfigure}
\usepackage{latexsym}
\usepackage{eurosym}
\usepackage{verbatim}
\usepackage{epstopdf}         % if you want to use .eps and .jpeg in the same document (compile with pdflatex directly)
\usepackage{array}            % permet d'utiliser m{width} dans l'env tabular
\usepackage[english]{babel}
\usepackage{multirow}
\usepackage{algorithmic}
\usepackage{units}
\usepackage{nicefrac}
\usepackage{natbib} 
\usepackage{appendix}
%\usepackage{chapterbib}
\usepackage{booktabs}
\usepackage{pdfpages}        % to include pdf pages into the doc \includepdf[pages={1}]{myfile.pdf}
\usepackage{layout}          % then \layout in the document to get the layout numbers
\usepackage{hyperref}        % Hyper references for the document
\hypersetup{
  colorlinks=true,       % false: boxed links; true: colored links
  breaklinks=true,    % permet le retour à la ligne dans les liens trop longs
  linkcolor=black,       % color of internal links (red)
  citecolor=black,       % color of links to bibliography (green)
  filecolor=black,       % color of file links (magenta)
  urlcolor=black,        % color of external links (cyan)
  pdftitle={\FileTitle},
  pdfauthor={\FileAuthor},
  pdfsubject={\FileSubject},
  pdfkeywords={\FileKeyWords}}

% BEAMER specific
\mode<article>{\usepackage{fullpage}}
\mode<presentation>
{
  \usetheme{default}
  \usecolortheme{default} % beetle, lily, beaver
}
\usepackage{tikz}
\usepackage{mdwlist}
\usepackage{moreverb}
\usepackage{etex}   % this is a dodgy fix for using lpic: http://www.tex.ac.uk/cgi-bin/texfaq2html?label=noroom
\usepackage{lpic}
\usepackage[absolute,overlay]{textpos} % for absolute positioning on page
\setbeamertemplate{navigation symbols}{}
\setbeamertemplate{footline}{\leavevmode% 
  \hfill\hbox{% 
    \begin{beamercolorbox}[wd=.15\paperwidth,ht=2.25ex,dp=1ex,right]{date in head/foot}% 
      \scriptsize \insertframenumber{} \hspace*{1ex} 
    \end{beamercolorbox}}% 
  \vskip0pt%
}
\beamertemplatetransparentcovereddynamic
\beamertemplateballitem
\beamertemplatesolidbuttons
\usefonttheme[onlymath]{serif}
\setbeamercovered{invisible} % http://tex.stackexchange.com/questions/53860/can-i-tell-beamer-that-uncover-should-be-invisible-not-merely-grayed-out


%=================================================================================
% Useful commands
\newcommand{\minimum}[1]{\mathop{\textrm{min}}_{#1}\,}
\newcommand{\maximum}[1]{\mathop{\textrm{max}}_{#1}\,}
\newcommand{\argmin}[1]{\mathop{\textrm{arg min}}_{#1}\,}
\newcommand{\dg}{\textsc{dg}}
\newcommand{\bs}{\boldsymbol}
\newcommand{\mbf}[1]{\mathbf{#1}}
\newcommand{\com}[1]{\textcolor{red}{#1}\marginpar{\textcolor{red}{\Large $/!\backslash$}}}
\newcommand{\pfrac}[2]{\frac{\partial#1}{\partial#2}}
\newcommand{\dpfrac}[2]{\dfrac{\partial#1}{\partial#2}}
\newcommand{\ufrac}[2]{\frac{\ud{}#1}{\ud{}#2}}
\newcommand{\dufrac}[2]{\dfrac{\ud{}#1}{\ud{}#2}}
\newcommand{\wt}[1]{\widetilde{#1}}
\newcommand{\avg}[1]{\left\{\!\!\left\{#1\right\}\!\!\right\}}
\newcommand{\jmp}[1]{\left[\!\left[#1\right]\!\right]}
\newcommand{\hr}{\textsc{hr}}
\newcommand{\xjl}{x_{j-\nicefrac{1}{2}}}
\newcommand{\xjr}{x_{j+\nicefrac{1}{2}}}
\newcommand{\xRl}{x_{I-\nicefrac{1}{2}}}
\newcommand{\xRr}{x_{I+\nicefrac{1}{2}}}
\newcommand{\xLl}{x_{(I-1)-\nicefrac{1}{2}}}
\newcommand{\xLr}{x_{(I-1)+\nicefrac{1}{2}}}
\newcommand{\ud}{\,\mathrm{d}}
\newcommand{\dgm}{\textsc{dgm}}
\newcommand{\fem}{\textsc{fem}}
\newcommand{\fvm}{\textsc{fvm}}
\newcommand{\cpu}{\textsc{cpu}}
\newcommand{\gpu}{\textsc{gpu}}
\newcommand{\gpub}{\textsc{gpublas}}
\newcommand{\blas}{\textsc{blas}}
\newcommand{\cublas}{\textsc{cublas}}
\newcommand{\cuda}{\textsc{cuda}}
\newcommand{\pcpu}{$P_{\cpu{}}$}
\newcommand{\pgpu}{$P_{\gpu{}}$}
\newcommand{\pgpub}{$P_{\gpub{}}$}
\newcommand{\pcpuc}{\textcolor{red}{$P_{\cpu{}}$}}
\newcommand{\pgpuc}{\textcolor{olivegreen}{$P_{\gpu{}}$}}
\newcommand{\pgpubc}{\textcolor{blue}{$P_{\gpub{}}$}}
\newcommand{\bth}{$B_{\text{theo}}$}
\newcommand{\bpr}{$B_{\text{pract}}$}
\newcommand{\bcpu}{$B_{\text{\cpu{}}}$}
\newcommand{\bgpu}{$B_{\text{\gpu{}}}$}
\newcommand{\bgpub}{$B_{\text{\gpub{}}}$}
\newcommand{\e}[1]{\ensuremath{\times 10^{#1}}}
\newcommand{\tabcn}[1]{\multicolumn{1}{c}{\makebox[0.65cm]{\tiny #1}}} % for a tabular workaround
\renewcommand*{\thesubfigure}{}  % Gets rid of the subfigure counter!!
\newcommand{\alf}{Alfvén}
\newcommand{\els}{Elsässer}
% Needed for the title page:
\newcommand{\HRule}{\rule{\linewidth}{0.5mm}}
\renewcommand{\topfraction}{0.85}
\renewcommand{\textfraction}{0.1}
\renewcommand{\floatpagefraction}{0.75}
% Needed for the natbib bibliography style
%\newcommand*{\newblock}{}
%% \bibpunct[<optional>]{}{}{;}{}{}{} 
% To put a box around a figure
\setlength\fboxsep{0pt}
\setlength\fboxrule{0.5pt}
\def\CC{{C\nolinebreak[4]\hspace{-.05em}\raisebox{.4ex}{\tiny\bf ++}}}

% increase space in underbrace env (http://tex.stackexchange.com/questions/13843/vertical-spacing-with-underbrace-command)
\newcommand*\mystrut[1]{\vrule width0pt height0pt depth#1\relax}

%=================================================================================
% Define some colors
\definecolor{violetred}{rgb}{0.78,0.08,0.52}
\definecolor{olivegreen}{rgb}{0.2,0.6,0.0}
\definecolor{darkgray}{rgb}{0.95,0.95,0.95}
\definecolor{mypurple}{rgb}{0.76,0.06,0.76}
\definecolor{greenyellow}{rgb}{0.76,0.76,0.06}
\definecolor{gold}{HTML}{FFD700} 
\definecolor{IMnumber}{HTML}{AFB9DB}

% From http://www.google.com/url?sa=t&rct=j&q=&esrc=s&source=web&cd=1&ved=0CFcQFjAA&url=http%3A%2F%2Fwww.perceptualedge.com%2Farticles%2Fvisual_business_intelligence%2Frules_for_using_color.pdf&ei=TioYUPL8GMPl0QHhqICQAQ&usg=AFQjCNGA9t1pGae49wQ0DVhaSNcAk6oLyA&sig2=QuA9D9nierTbqn3djCyyPQ
\definecolor{c1med}{HTML}{F15A60} % red
\definecolor{c2med}{HTML}{7AC36A} % green
\definecolor{c3med}{HTML}{5A9BD4} % blue
\definecolor{c4med}{HTML}{FAA75B} % orange
\definecolor{c5med}{HTML}{9E67AB} % purple
\definecolor{c6med}{HTML}{CE7058} % burgundy
\definecolor{c7med}{HTML}{D77FB4} % magenta
\definecolor{c8med}{HTML}{737373} % grey

\definecolor{c1brt}{HTML}{EE2E2F} % red     
\definecolor{c2brt}{HTML}{008C48} % green   
\definecolor{c3brt}{HTML}{185AA9} % blue    
\definecolor{c4brt}{HTML}{F47D23} % orange  
\definecolor{c5brt}{HTML}{662C91} % purple  
\definecolor{c6brt}{HTML}{A21D21} % burgundy
\definecolor{c7brt}{HTML}{B43894} % magenta 
\definecolor{c8brt}{HTML}{010202} % black

\newcommand{\tcb}[2]{\textcolor{c#1brt}{#2}}
\newcommand{\tcm}[2]{\textcolor{c#1med}{#2}}

%=================================================================================
% Tikz special stuff (do after color definitions)
\usetikzlibrary{decorations.pathreplacing}
\usetikzlibrary{calc}
\usetikzlibrary{decorations.shapes}
\usetikzlibrary{decorations.pathmorphing}
\tikzset{paint/.style={fill=red}, decorate with/.style={decorate,decoration={shape backgrounds,shape=#1,shape size=3pt}}}
\tikzset{dotr/.style={fill=c1brt,circle,minimum size=3pt}}
\tikzset{dotb/.style={fill=c3brt,circle,minimum size=3pt}}
\tikzset{dotrr/.style={fill=c1brt,circle,minimum size=0.2cm,inner sep=0}}
\usetikzlibrary{positioning}
\usetikzlibrary{shapes}
\usetikzlibrary{snakes}
%fill=red,circle,minimum size=3pt
\usepackage{pgfkeys}
\usepackage{pgf,pgfarrows,pgfnodes,pgfautomata,pgfheaps,pgfshade}

%=================================================================================
% Environments 
% Theorem environments (from http://www.math.uiuc.edu/~hildebr/tex/theorems.html)
\newtheorem{thm}{Théorème}[section]
\newtheorem{cor}[thm]{Corollary}
\newtheorem{lem}[thm]{Lemma}
%\theoremstyle{remark}
\newtheorem{rem}[thm]{Remark}
%\theoremstyle{definition}
%\newtheorem{def}[thm]{Definition}

% Tight lists
\newenvironment{tight-itemize}
{\begin{itemize}
  \setlength{\itemsep}{1pt}
  \setlength{\parskip}{0pt}
  \setlength{\parsep}{0pt}}
{\end{itemize}}
\newenvironment{tight-enumerate}
{\begin{enumerate}
  \setlength{\itemsep}{1pt}
  \setlength{\parskip}{0pt}
  \setlength{\parsep}{0pt}}
{\end{enumerate}}
\newenvironment{tight-description}
{\begin{description}
  \setlength{\itemsep}{1pt}
  \setlength{\parskip}{0pt}
  \setlength{\parsep}{0pt}}
{\end{description}}

%=================================================================================
% For code inclusion
\usepackage{listings}
\lstset{
language=c,                        % choose the language of the code
basicstyle=\tiny,                  % the size of the fonts that are used for the code
backgroundcolor=\color{white}, %\color{darkgray},  % choose the background color. You must add \usepackage{color}
showspaces=false,               % show spaces adding particular underscores
showstringspaces=false,         % underline spaces within strings
showtabs=false,                 % show tabs within strings adding particular underscores
frame=none,                     % adds a frame around the code
tabsize=1,                      % sets default tabsize to 4 spaces
captionpos=t,                   % sets the caption-position to bottom
breaklines=true,                % sets automatic line breaking
numbers=none,                   % where to put the line-numbers
numberstyle=\tiny,              % the size of the fonts that are used for the line-numbers
stepnumber=1,                   % the step between two line-numbers. If it's 1 each line will be numbered
numbersep=5pt,                  % how far the line-numbers are from the code
keywordstyle=\color[rgb]{0,0,1},
commentstyle=\color[rgb]{0.133,0.545,0.133},
stringstyle=\color[rgb]{0.627,0.126,0.941},
morekeywords={scalar,dim3, end}      % if you want to add more keywords to the set
}

%=================================================================================
% For media inclusion
%\usepackage{movie15}
%\usepackage{media9}
\usepackage{multimedia}
